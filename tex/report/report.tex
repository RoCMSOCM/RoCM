\documentclass[titlepage]{article}

\usepackage[english]{babel}
\usepackage{pdfpages}\usepackage{amsmath}

\usepackage[top=1.5in, bottom=1.5in, left=2in, right=2in]{geometry}

\newcommand{\myparagraph}[1]{\paragraph{#1}\mbox{}\\}


\addtolength{\oddsidemargin}{-.875in}
\addtolength{\evensidemargin}{-.875in}
\addtolength{\textwidth}{1.75in}

\begin{document}

\title{
\textbf{
Models of the Rotation Curve}
\protect\\
for the
\protect\\
\textbf{
Milky Way Galaxy}}

\author{Robert Moss, Dr. James G. O'Brien}
\maketitle

\newpage
\tableofcontents{} 
\newpage

\section{Introduction}
When asked the question, ``Name a the first galaxy that comes to mind?" I would suspect almost everyone would answer with ``The Milky Way" and not with ``NGC 2403". That answer is the reason I wanted to model the Milky Way galaxy because our neighborhood of star clusters hold so much information.

\section{Scope}
%Scope of your project: Touch on CS major, not physics major, and details of work such as data mining, contacting of authors, research of methods, modeling (including parameters such as disk, gas, bulge etc) and CS fitting.
Physics -- for me -- is more of a passion than a profession. I'm about to graduate with a computer science degree and a minor in physics. While most of you know, astrophysics is dependent on observations and relies heavily on computation models to further test and research. My goal in this research is to apply my computation knowledge in the realm of physics. I've created a tool which uses efficient technology and seamless interfacing to not only help myself along the way, but to aid future research in the rotation curves of every known galaxy.

\section{History}
%Brief History of rotation curves and Dark Matter.
%As early as 1932, Jan Oort noticed that the interstellar stars were moving faster than expected (Jan Oort, 1932). 
In 1933, Fritz Zwicky working for the \textit{California Institute of Technology} observed the red shift of star clusters within galaxies and stated that the expected velocity was entirely off. The purposed theory was that a matter invisible to the eye (dark matter) was in need to account for missing mass that would otherwise hold the clusters together (Zwicky). He proposed that there were ``faint galaxies and diffused gas" that filled in this void (Zwicky). It was later confirmed that such gas was indeed within the galaxy, and as telescopes moved from optical to radio telescopes, the fidelity was high when measuring the amount of gas thanks to the success of spectroscopy.  Although we can now account for this ``missing mass'', many years of evidence has shown that the amount from gas was no where near the volume that was missing in the overall observation. In 1970, Vera Rubin and her colleague, Kent Ford, worked with a new spectrograph that could observe the rotation velocity of spiral-galaxies with incredible accuracy; in both the optical and radio bands (Rubin, Ford 1980). Upon further research, they concluded that the stars were moving with a uniform velocity around the center of the galaxies; rather than gradually declining further out (as predicted in Einstein's equation).  Moreover, as their observations advanced, the uniform velocity seemed to trend across all sizes and shapes of galaxies, which would rule out effects of star formation and or galactic evolution. Their findings brought along much speculation.  Since at its heart, the predicted velocity $v$ is a function of only two observational parameters, the distance from the center and the mass. It was realized that if there was some invisible mass in the outer regions of galaxies, the prediction and data could be reconciled. This missing factor that increased the predicted velocity is highly thought to be dark matter. The research I've conducted takes Rubin's understanding of the galactic rotation curve, a few other alternate dark matter approaches, and applies it to a model of the Milky Way galaxy.  This is an important study since until recently (over the last ten years) studies of the rotation curve of the Milky Way galaxy was limited, but now is explored.

\section{Data Gathering}
First, I started to collect data about the Milky Way. I found incredible resources through \textit{arxiv} and Harvard's \textit{Astrophysics Data System}. I reached out to Yoshiaki Sofue, at the \textit{Department of Physics and Astronomy of Kagoshima University} in Kagoshima, Japan. I received 572 data points of the rotation velocity ranging from 0.09 kpc to 20 kpc. Not being satisfied enough with this range, I hunted for more. I came across a paper that had data for the rotation curve of the Milky Way all the way out to about 200 kpc. I contacted the authors who were working for the \textit{Saha Institute of Nuclear Physics} in Kolkata, India (Bhattacharjee, Chaudhury, Kundu). They graciously sent me their 50 data points. I now had a clear, synthesized model model of the Milky Way's rotation curve.  Before synthesizing the data together we had to be sure that the data collected from the various sources were consistent and later analyze the viability of such data.  We decided to take all the data from Sofue as well as the data from Kundu et all out to about 100kpc since these are the most well documented sources.

\section{Computation}

\subsection{Julia and D3.js}
I was now posed with the question on how to develop this computation model of the galaxy. While working at MIT Lincoln Laboratory, I got the pleasure of working closely with Jeff Bezanson, the co-creator of the programming language \textit{Julia}. Jeff showed me how useful Julia truly was when it came to mathematical computation and data manipulation. I personally chose Julia for several reasons; it's fast, comparable to C, and it can be easily parallelized.

Julia handled all the heavy computation, while a JavaScript interface primarily using the library D3 (Data Driven Documents) was used to plot the results. The flexibility of this interface allows for any galaxy's rotation curve to be inputted and the represented curves can be manipulated and experimented on. Sliders are present for the values deemed as ``free parameters''.

\subsection{Data Plot}



\subsubsection{Einstein's General Relativity Prediction}

For Einstein's curve, I used a $\chi^2$ test to find the optimal number of stars for the Milky Way, namely $4.4*10^{10}$. Since this value varies across research, we wanted to test to see what value came about from our model. The number found from the {$\chi^2$} fit is consistent in the known established number of star range for the milky way galaxy.


\begin{equation}
v_{\text{lum}} = \sqrt{\frac{N^*\beta^*c^2R^2}{2R^3_0}\left[I_0\left(\frac{R}{2R_0}\right)K_0\left(\frac{R}{2R_0}\right)-I_1\left(\frac{R}{2R_0}\right)K_1\left(\frac{R}{2R_0}\right)\right]}
\end{equation}

As you can see, the prediction after about 18 kpc falls flat. This is where the dark matter contribution would most likely start to effect the total curve. All spiral galaxies have the famous curve represented here. This curve is the main reason for dark matter and poses the question of the missing velocity.

The bulge contribution was added in order to fully model the galaxy. Similarly, the gas contribution was also added. As Zwicky first thought, the gas contribution only increases the overall rotation velocity slightly, deeming that it's not the reason for the missing velocity in the outer regions.
 
\subsection{Dark Matter Fitting}
%Showing the fitting of the galaxy with : "known parameters" and discussing the impact and overall estimate of the dark matter.  Showing the galaxy as to what it "wants" with free parameters for number of stars and dark matter.

The dark matter formulation has two free parameters, $\sigma_0$ and $r_0$. Using a $\chi^2$ test against the observed data, we find each free parameter. In general, this fitting has to be done for each galaxy, as there is no general dark matter equation for any given galaxy. I studied how the change of $\sigma_0$ and $r_0$ effected the overall rotation curve. 


\begin{equation}
v_{dark} = \sqrt{4\pi\beta^*c^2\sigma_0\left[1-\frac{r_0}{R}\text{arctan}\left(\frac{R}{r_0}\right)\right]}
\end{equation}

\begin{equation}
\text{Free parameter}\;\sigma_0 = \text{spherical dark matter density}
\end{equation} % isothermal sphere

\begin{equation}
\text{Free parameter}\;r_0 = \text{core radius or halo radius}
\end{equation}

\subsubsection{Dark Matter Contribution}

As you can see, the dark matter curve doesn't start to contribute until about the 10 kpc.

\subsubsection{Total Expected Curve}
I summed the curve generated by Einstein's rotational velocity equation and the dark matter fitting to produce the overall rotation curve for the Milky Way galaxy.

This curve fits the data perfectly, as expected. Since the dark matter contribution was formed around the difference in the data and the General Relativity curve, it's obvious that we would see such a result. The issue here is that this isn't good science. Having two free unknowns in an equation that has to be fitted to the data sparks a red flag. I'll cover a few alternate theories, and discuss the work my adviser, Dr. James G. O'Brien, has been developing.

\section{Alternate Methods}
%Talking about the later points, and how they are remaining flat or declining which will lead you into a discussion of the current dark matter picture vs alternative methods

\subsection{Modified Newtonian Dynamics (MoND)}
Mordehai Milgrom's theory stated the Newtonian laws needed to me changed. An initial interpretation of MoND explains that Newton's second law of motion needed to be altered, but later ran into conflicts with the conservation of momentum (MoND, 1983). Another interpretation of this theory is the change of the law of gravity. The law of gravity needs to not only be affected by the mass $m$, but by $\frac{m}{\mu(a,a_0)}$. The function $\mu$ takes the acceleration $a$ and the natural constant $a_0 \approx 10^{-10} \frac{m}{s^2}$. Because this theory won't hold in small mass and small acceleration experiments, this seems to be the most understandable modification.

Part of the issue with MoND is that it was formulated to solve the rotation curve problem and not derived from any known action principle. MoND is an ad hoc theory just like dark matter. MoND doesn't contain any explicit free parameters, but the modified dynamics has to be calculated depending on the class of the galaxy; namely, small or large galaxies. Meaning there is a universality that MoND is still missing. MoND alters Newtonian physics (which works well on the human sized scale) but a scalable answer needs to instead modify Einstein's theory of general relativity. This is where conformal gravity comes into place.


\subsection{Conformal Gravity}

% A brief discussion of why the conformal curve fits, and how it differs from cold dark matter.
Starting with the principle equations of general relativity, conformal gravity deems to replace GR, but still allows GR to stay true under certain scales. The intent of GR wasn't for galaxies and the universe as a whole, hence it works perfectly for solar systems, because that's exactly what it was meant for. The motivation for general relativity's discovery was the fact that we wanted to predict the movement of the planets.  GR replaced Newtonian physics, yet it still allowed Newtons laws to remain true under certain scales. For the same motivation as general relativity, conformal gravity encompasses the scales of galaxies and the universe to explain the interactions between these astronomical objects. 

Could we formulate an equally good theory of gravity that is more inclusive than general relativity? Yes, conformal gravity starts from the first principles of general relativity, and obeys them throughout. Three additional terms to the GR equation have been appended, $\gamma^*$, $\gamma_0$, and $\kappa$. These parameters include missing feasible physical parameters that Einstein didn't need in order for him to model the solar system. Now that we need to model objects at a greater scale, the parameters that were initially scrapped from the theory of general relativity have now been added back in.

A scalar action has to be derived from a field in order to model gravity. General relativity's scalar action is in the form 

\begin{equation}
S_{GR} = \int d^4x \sqrt{-g} R_{GR}.
\end{equation}

Not created to solve the rotation curve problem, conformal gravity modifies the scalar action of general relativity.

\begin{equation}
S_{CG} = \int d^4x \sqrt{-g} C,
\end{equation} 
\begin{equation}
C \; \alpha \; R_{GR} + \gamma^* + \gamma_0 - \kappa.
\end{equation}

The parameters exist only at a certain scale. This allows conformal gravity to scale down to the solar system, making the three additional small terms negligible. Similarly, general relativity can scale down to obey Newtonian physics. By transitivity, conformal gravity encompasses not only general relativity, but Newtonian laws of motion as well. The equation for the scalar action of conformal gravity is

\begin{equation}
S_{CG} \; \alpha \; S_{GR} + \gamma^* + \gamma_0 - \kappa.
\end{equation}

The three terms are related by this equality: $\kappa < \gamma_0 < \gamma^*$. The first term, $\gamma^*$, is a matter inclusive term. Accounting for missing matter in the general relativity formulation, this local matter term has a similar strength to the general relativity term. A global term, $\gamma_0$, includes the matter from the rest of the universe. This linear factor calculates the pull of other galaxies in the universe, because a uniformly separated universe cannot be assumed. The last parameter, and the smallest, $\kappa$ is an inhomogeneity term. The overestimation of matter within the galaxy can be thought of as counting mass that is not within the localized collection of matter, in our case the galaxy.



\begin{equation}
v_{\text{CG}} =
\end{equation}
\begin{equation*}
\sqrt{\genfrac{}{}{0pt}{}{\frac{N^*\beta^*c^2R^2}{2R^3_0}\left[I_0\left(\frac{R}{2R_0}\right)K_0\left(\frac{R}{2R_0}\right)-I_1\left(\frac{R}{2R_0}\right)K_1\left(\frac{R}{2R_0}\right)\right]}{+\frac{N^*\gamma^*c^2R^2}{2R_0}I_1\left(\frac{R}{2R_0}\right)K_1\left(\frac{R}{2R_0}\right)+\frac{\gamma_0c^2R}{2}-\kappa c^2R}}
\end{equation*}

\section{Demonstration}
%A showcase of RoCM, and the animations to discuss the value they add in discussion, understandability and dissemination of the information contained in a rotation curve.

\subsection{Rotational Curve Modeler (RoCM)}
My main goal was to help generalize the work being done on galaxy research. Researches will all need to model the investigated galaxy either in MATLAB or Mathematica, but there doesn't exist a singular tool to expedite this process in a general format. With raw data as the single input, any galaxy can be imported into the tool, RoCM. The tool can plot the data graph, and include several curves as overlays. Aiding in finding alternate theories to dark matter, the purpose of RoCM is to be able to adjust the free parameters within the dark matter equation. The scientist or researcher working with this tool can visually see how each free parameter effects the curve. It has the built in functions for the Einstein rotation curve (including both the optical luminous matter as well as gas matter), the dark matter curve, conformal gravity, and includes the bulge contribution. The user can also export the graph in an SVG format. A Scalable Vector Graphic is a loss-less format and is particularly important for data visualization.  It is a later goal of RoCM to be able to be used not only for the milky way galaxy, but to be used universally for any given galaxy.

%Highlighting the flexibility of your code.  Where we can go from a previously done fit to one on the fly, and talk about the power of such a tool for this galaxy as well as for an arbitrary galaxy.

\section{Conclusion}
I wanted to present an unbiased approach to modeling the rotation curve of the Milky Way galaxy. I sought out to understand \textit{how} and \textit{why} each theory was derived, namely $\Lambda$CDM, MoND, and Conformal Gravity. In doing so I've accepted each theory as truth in order to effectively model each set of equations. Along side my research, I built a tool to help myself and other physicists model the rotation curve of any given galaxy; providing several different theories to model. My hope was to surface several outstanding questions that are still in need of debate. %Thank you.


\end{document}